\section{Regression Analysis}
\label{sec:regression_analysis}

For our regression analysis we built four different specifications to best capture the 
relationship between the distance to a subway station and the log of price. We choose a log-linear model so that our coefficients
are interpretable as percentage changes in price. We inspect our variable of interest in two forms, continuous so we can 
see the marginal effect of each additional meter of distance on price, and as a dummy variable to see the effect of being subjectively close to a subway station.
We look at Figure~\ref{log_price_scatter} to see the relationship between distance and price, and decide on 2000m as distinguishing being close to a station or far.
For our analysis, we assume that our data is identical and independently distributed since we do not expect the lease transfer of one house to affect the lease transfer of another.
In our data cleaning section we windsorize distance at 10km and can assume our data contains no large outliers.
We use robust standard errors in our regressions, and therefore aren't concerned about the homoscedasticity of errors.

\subsection{Univariate Linear Regressions}
\subsubsection{Univariate Specifications}
\begin{equation}
   \log(\text{Price}) =\beta_0 + \beta_1 \cdot \text{Min\_dist}+\epsilon
      \label{eq:univariate_linear_regression}
\end{equation}
\begin{equation}
   \log(\text{Price}) =\beta_0+\beta_2 \cdot \text{close}+\epsilon
   \label{eq:dummy_univariate_linear_regression}
\end{equation}

Where:
\begin{itemize}
   \setlength{\itemsep}{0pt}%
   \setlength{\parskip}{0pt}%
   \setlength{\parsep}{0pt}%
      \item $Log(Price)$ is the log of the price of the house
      \item $\beta_0$ is the intercept and not shared between the two specifications
      \item $\beta_1$ is the coefficient of the distance to the nearest subway station
      \item $\beta_2$ is the coefficient of the distance dummy, whether the house is close (within 2000m) to a subway station or not
      \item $\text{Min\_dist}$ is the distance to the nearest subway station
      \item $\epsilon$ is the error term
   \end{itemize}

\subsubsection{Univariate Results}
We start with our simplest specifications. We regress the log of price on the distance to the nearest subway station to get a basic understanding of the 
relationship between our variable of interest and our dependent variable. 
Using this specification we find statistically significant null effects for distance on price . 
Table~\ref{summary_regression_table} shows us that the coefficient of our variable of interest is -.0000431 at above the 5\% significance.
The coefficient can be interpreted as a 0.0000431\% decrease in house price for each additional meter of distance from a subway station, which is economically insignificant.
 Even houses positioned 10 km away from a subway station wouldn't even observe a difference above 1\% in house price from the mean.
Looking at the results from specification \ref{eq:dummy_univariate_linear_regression} we see that the coefficient is economically and statistically significant,
with a coefficient representing a 0.31\% increase in house price for houses within 2000m of a subway station (Table~\ref{summary_regression_table}).
What this tells us is that the marginal effect of being a metre closer to a subway station is not significant. But, buyers' willingness to pay follows a piecewise function, and have some threshold of distance over which
they are willing to pay more for a house. Essentially, above a certain threshold the distance from a subway station is not a consideration.
In this case, whether you are within 2km of a subway station or not is a significant factor in determining house price, and adds a 0.33\% premium.

However, these models are underspecified. They suffer from omitted variable bias, there are variables that comove with distance from a subway station that also affect house prices, and we delve into these
controls in our multivariate analysis. We would expect the economic significance of distnace to increase with the inclusion of these controls.
Since subway stations are typically central, we would expect neighbourhoods further away to have lower house prices. Omitting 
district controls would bias our coefficient to be more positive than it should be. We would also expect house prices to increase over time, whereas they decrease with min\_dist. Omitting time controls would bias our coefficient to be more 
positive than it should be.
\subsection{Multivariate Linear Regressions}

\subsubsection{Multivariate Specifications}
\begin{equation}
   \begin{aligned}
   \log(\text{Price}) &= \beta_0 + \beta_1 \cdot \text{Min\_dist} + \beta_3 \cdot (\text{Min\_dist})^2 \\
   &\quad + \beta_4 \cdot \text{oldnew} + \beta_5 \cdot \text{district}
   + \beta_6 \cdot \text{year} + \beta_7 \cdot (\text{district} \times \text{year}) + \epsilon
   \end{aligned}
   \label{eq:multivariate_linear_regression}
\end{equation}
   
\begin{equation}
   \begin{aligned}
   \log(\text{Price}) &= \beta_0 + \beta_2 \cdot \text{close}\\
   &\quad + \beta_4 \cdot \text{oldnew} + \beta_5 \cdot \text{district} 
   + \beta_6 \cdot \text{year} + \beta_7 \cdot (\text{district} \times \text{year}) 
   + \epsilon
   \end{aligned}
   \label{eq:dummy_multivariate_linear_regression}
\end{equation}

Where:
\begin{itemize}
   \setlength{\itemsep}{0pt}%
   \setlength{\parskip}{0pt}%
   \setlength{\parsep}{0pt}%
      \item $\beta_3$ is the coefficient of the squared distance to the nearest subway station
      \item $\beta_4$ is the coefficient on the age of the house
      \item $\beta_5$ is a matrix of coefficients for district dummies
      \item $\beta_6$ is the coefficient of the year
      \item $\beta_7$ is a matrix of coefficients for the interaction of district and year dummies
      \item $\text{oldnew}$ is a dummy variable for the age of the house
      \item $\text{district}$ represents dummy variables for the district of the house
      \item $\text{year}$ represents dummy variables for the year of the observation
   \end{itemize}

We include a squared term for distance in our multivariate regression to account for the non-linear relationship between distance and price we find in Figure\ref{log_price_scatter}.

We now include controls for the age of the house, the district of the house, and the year of the observation. The economic basis for this, is we expect there to be 
price variation between older and newer homes because of the quality of the home. We also expect price variation between districts as some districts
hold premiums due to proximity to a city center or other amenities. We also expect price variation over time as the economy grows and inflation occurs.
Consider that some districts receive more or less government funding and expansion. Since we might expect the time varying effect to be different across districts, we include an interaction term between district and year.

\subsubsection{Multivariate Results}
Table~\ref{tab:summary_regression_table} hides the coefficients of the controls, but a full table of the results can be found in the appendix under Table~\ref{tab:full_MV_regression_results} and Table~\ref{tab:full_MV_regression_results_w_dummy}.

In our multivariate regression with the continuous polynomial for distance (Specification \ref{eq:multivariate_linear_regression}), we find that the coefficient is still economically insignificant (-0.0001261). As expected, controlling for district and time-varying effects increases the magnitude of our result, 
but we still conclude a precise null marginal effect of distance on house price.

In our final specification (Specification~\ref{eq:dummy_multivariate_linear_regression}),
 we find that the coefficient of the dummy variable for being close to a subway station is still economically and statistically significant. 
 However, much of the variation is now explained by the controls, and we find a more reasonable 0.18\% increase attributable to being close (Within 2km) to a subway station.

Notably, our $R^2$ jumps from 0.03 in our univariate regression to 0.6 in our multivariate regression(Table \ref{tab:full_simple_regression_results} and Table~\ref{tab:full_MV_regression_results}).
This is a good sign that our multivariate specifications which include controls are explaining much of the variation in logarithmic house prices.

Looking at the predicted versus actual plots in Figure~\ref{MV_bucketed_predicted_vs_actual} and Figure~\ref{MV_dummy_bucketed_predicted_vs_actual}, we see that our models are well-specified, apart from at the lower tail.
Most predicted values lie on the 45 degree line, however, we are systematically underpredicting the prices of the lowest priced homes. This is further reflected in the residual fits
as seen in Figure \ref{MV_fitted_vs_residuals} and Figure \ref{MV_dummy_fitted_vs_residuals}. We see that our residuals are randomly distributed around 0 apart from at the lower tail, where they are systematically positive.