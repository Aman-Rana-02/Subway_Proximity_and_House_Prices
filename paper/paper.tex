\documentclass[11pt,a4paper,english]{article}
\usepackage{natbib}  % Adds support for different citation styles
\bibliographystyle{chicago}
\usepackage[T1]{fontenc}
\usepackage[utf8]{inputenc}
\usepackage{babel}
\usepackage{blindtext}
\usepackage[nodayofweek,level]{datetime}
\newdate{date}{05}{12}{2024}
\date{\displaydate{date}}
\usepackage[a4paper,margin=1in]{geometry}
\usepackage{graphicx}
\usepackage{setspace}
\usepackage{amsmath}
\usepackage{hyperref}
\usepackage{tabularx}
\usepackage{booktabs}
\usepackage{longtable}
\doublespacing



\title{Analyzing the effects of Subway proximity on real estate prices in London 
\thanks{Code and data supporting this analysis is available at: \url{https://github.com/Aman-Rana-02/Subway_Proximity_and_House_Prices}}}
\author{%
  Aman Rana \\ 1007034692 \\ aman.rana@mail.utoronto.ca
  \and
  Pierre Sarrailh \\ 1006770111 \\ pierre.sarrailh@mail.utoronto.ca
}
\date{}
\begin{document}
  \maketitle
  \begin{center}
    ECO375: Applied Econometrics \\ 
    University of Toronto \\ 
    Department of Economics \\
    5\textsuperscript{th} December, 2024
  \end{center}

  \begin{abstract}
    \noindent Following \citet{zhou_2019} we saw that the completion of subway lines increased the prices of real estate in their proximity. Our contribution will be to look at these effects in London. 
     We found precise null effects by a multivariate linear regression of distance from a station on the log of house prices, with controls for time and district. In a regression of log price on the distance 
     from the nearest station with time and district controls, the slope coefficient is -0.0000268 with a standard error of 0.000000687, 
     a statistically significant relationship of such small effect it led us to conclude precise null effects. 
     The effect is economically insignificant, even a 10km distance from a station (which is above the 95th 
     percentile of our distance data) would lead to only a 0.2\% decrease in house price. 
     On more relative terms, a standard deviation difference in distance is only associated with a 
     14\% of a standard deviation change in log house price. 
     A possible further investigation would be to look at the treatment effect of having a 
     nearer subway station using a difference-in-difference (DiD) framework. 
     Using the DiD framework we would be aiming to investigate the effect of specifically 
     new subway stations on house prices while accounting standardizing over previously built ones.
  \end{abstract}

  % \newpage
  % \tableofcontents

  \newpage
  \section{Introduction}
The relationship between transport infrastructure and real estate prices has been a topic of interest in econometrics for many years now. Accessibility to public transit has been long established as a key factor influencing the price of residential properties, particularly in dense urban environments where vehicle mobility is greatly constrained.
 In the context of the Greater London Area subway stations are the most efficient method of transportation for most commuters and thereby play a critical role in shaping the greater housing market.
  To this end in this paper we investigate how the euclidean distance from a residence to its nearest subway station affects market price 

By focusing on the Greater London Area our study contributes to already established relationships between subway proximity and house prices. 
Early studies such as those done by \citet{muth_1969} are cornerstones in establishing a relationship between densification and land use in cities. 
More recently a greater number of studies have found a significant positive relationship between transit proximity and real estate prices 
in cities such as those in Shanghai \citep{zhou_2019} where subway stations proximity was specifically correlated to 
increased house prices at statistically significant results. 
However London presents unique opportunities for research. 
As the historical pioneer in subway technology the city has had 
the opportunity to develop around an already established subway system and culture. 
This study extends the findings of existing literature by using highly specific spatial data and 
leveraging advanced analysis techniques to create clear and insightful findings on the pricing dynamics of the London underground.


The rest of the paper is structured as follows: Section \ref{sec:data} describes the data used in this analysis. Section \ref{sec:regression_analysis} outlines the methodology and model specifications used to measure the subway effect.
Section \ref{sec:results} presents the results of our analysis. Section \ref{sec:conclusion} discusses the limitations of our methodology and concludes. Appendix \ref{sec:appendix} holds figures and tables.


  \section{Data}
\label{sec:data}
We began our analysis by sourcing data from ONS, the national statistics arm of the UK government,
 which provided panel data containing each lease transfer (home sale) that has occurred in the U.K. from 1995 to today \citep{land_ukgov}.
  We pruned this dataset to only contain observations within the Greater London Area.
   This dataset amounts to approximately 350 thousand observations. Each unit of observation represents
    a single occurrence of a lease transfer in London containing the date, coordinates, neighborhood, and price of each observation's
     lease transfer.  The second source of our data was a table containing the coordinates of each subway station in the
      London underground and the date at which it was opened \citep{subway_wikipedia}. 
To begin our analysis we combined these two datasets by using the date and coordinates of our houses to search for the closest
 subway station to the current house in the second dataset. This allowed us to get the minimum distance to a subway station at
  the time of its sale which we added to our ONS dataset as the variable calculating the euclidean distance to the closest station
   and a variable for the coordinate of the closest subway. For all houses which were further than 10 km away from a subway station
    we set their distance to be equal to 10 km to account for outliers in our data.
Delving into the summary statistics of our data, we can look at how the data is spread
 across our different variables. For our main variables of interest, the distance to subway stations, we see that we have an
  average distance of 2821 meters with a variance of 2915 meters (Table~\ref{tab:continuous_summary_stats}). This means that on average houses are located quite far from
   a subway station but also contain vast amounts of variation. From our earlier work on the matter the reason for this high
    variation and mean is due to the fact that a lot of houses are located very close to subway stations but there is a very
     long tail in our data with a significant number of houses located over 5 kilometers away. We truncate these
     outliers at 10km, and the distribution of the minimum distance to a subway station can be seen in Figure~\ref{min_dist_distribution}.
     Raw house price is also right skewed with a long tail, so we take the log of house prices to make our regression more interpretable
      and the distribution of log price can be seen in Figure~\ref{log_price_distribution}.
On the categorical side of things we see that our houses are very evenly spread between our neighborhoods with the largest
 neighborhood only being 2 points off of the mean (Table~\ref{tab:summary_district}). Similarly, our lease transfers were also very evenly spread between years
  with each year accounting for approximately 2\% 
  of our dataset (Table~\ref{tab:summary_year}). This is good for the robustness of our future models since the even spread of data between 
  our neighbourhoods would provide us enough data to account for outliers within neighbourhoods. 
  On the other hand the spread between newly built and old homes is quite dismal with 97\%
   of lease transfers being accounted for by the sale of pre-owned homes and as a result it will be more sensitive to outliers (Table~\ref{tab:summary_ON}).


  \section{Regression Analysis}
\label{sec:regression_analysis}

For our regression analysis we build four different specifications to best capture the 
relationship between the distance to a subway station and the log of price. We choose a log-linear model so that our coefficients
are interpretable as percentage changes in price. We inspect our variable of interest in two forms, continuous so we can 
see the marginal effect of each additional meter of distance on price, and as a dummy variable to see the effect of being subjectively close to a subway station.
We look at Figure~\ref{log_price_scatter} to see the relationship between distance and price, and decide on 2000m as distinguishing being close to a station or far.
For our analysis, we assume that our data is identical and independently distributed since we do not expect the lease transfer of one house to affect the lease transfer of another.
In our data cleaning section we windsorize distance at 10km and can assume our data contains no large outliers.
We use robust standard errors in our regressions, and therefore aren't concerned about the homoscedasticity of errors.

\subsection{Univariate Linear Regressions}
\subsubsection{Univariate Specifications}
\begin{equation}
   \log(\text{Price}) =\beta_0 + \beta_1 \cdot \text{Min\_dist}+\epsilon
      \label{eq:univariate_linear_regression}
\end{equation}
\begin{equation}
   \log(\text{Price}) =\beta_0+\beta_2 \cdot \text{close}+\epsilon
   \label{eq:dummy_univariate_linear_regression}
\end{equation}

Where:
\begin{itemize}
   \setlength{\itemsep}{0pt}%
   \setlength{\parskip}{0pt}%
   \setlength{\parsep}{0pt}%
      \item $Log(Price)$ is the log of the price of the house
      \item $\beta_0$ is the intercept and not shared between the two specifications
      \item $\beta_1$ is the coefficient of the distance to the nearest subway station
      \item $\beta_2$ is the coefficient of the distance dummy, whether the house is close (within 2000m) to a subway station or not
      \item $\text{Min\_dist}$ is the distance to the nearest subway station
      \item $\epsilon$ is the error term
   \end{itemize}

\subsubsection{Univariate Results}
We start with our simplest specifications. We regress the log of price on distance to get a basic understanding of the 
relationship between our variable of interest and our dependent variable. 
Using this specification we find statistically significant null effects for distance on price. 
Table~\ref{tab:summary_regression_table} shows us that the coefficient of our variable of interest is -.0000431 at above the 5\% significance.
The coefficient can be interpreted as a 0.0000431\% decrease in house price for each additional meter of distance from a subway station, which is economically insignificant.
 Even houses positioned 10 km away from a subway station wouldn't even observe a difference above 1\% in house price from the mean.
Looking at the results from specification \ref{eq:dummy_univariate_linear_regression} we see that the coefficient is economically and statistically significant,
with a coefficient representing a 0.31\% increase in house price for houses within 2000m of a subway station (Table~\ref{tab:summary_regression_table}).
What this tells us is that the marginal effect of being a metre closer to a subway station is not significant. But, buyers' willingness to pay follows a piecewise function, and have some threshold of distance over which
they are willing to pay more for a house. Essentially, above a certain threshold the distance from a subway station is not a consideration.
In this case, whether you are within 2km of a subway station or not is a significant factor in determining house price, and adds a 0.33\% premium.

However, these models are underspecified. They suffer from omitted variable bias, there are variables that comove with distance from a subway station that also affect house prices, and we delve into these
controls in our multivariate analysis. We would expect the economic significance of distnace to increase with the inclusion of these controls.
Since subway stations are typically central, we would expect neighbourhoods further away to have lower house prices. Omitting 
district controls would bias our coefficient to be more positive than it should be. We would also expect house prices to increase over time, whereas they decrease with min\_dist. Omitting time controls would bias our coefficient to be more 
positive than it should be.
\subsection{Multivariate Linear Regressions}

\subsubsection{Multivariate Specifications}
\begin{equation}
   \begin{aligned}
   \log(\text{Price}) &= \beta_0 + \beta_1 \cdot \text{Min\_dist} + \beta_3 \cdot (\text{Min\_dist})^2 \\
   &\quad + \beta_4 \cdot \text{oldnew} + \beta_5 \cdot \text{district}
   + \beta_6 \cdot \text{year} + \beta_7 \cdot (\text{district} \times \text{year}) + \epsilon
   \end{aligned}
   \label{eq:multivariate_linear_regression}
\end{equation}
   
\begin{equation}
   \begin{aligned}
   \log(\text{Price}) &= \beta_0 + \beta_2 \cdot \text{close}\\
   &\quad + \beta_4 \cdot \text{oldnew} + \beta_5 \cdot \text{district} 
   + \beta_6 \cdot \text{year} + \beta_7 \cdot (\text{district} \times \text{year}) 
   + \epsilon
   \end{aligned}
   \label{eq:dummy_multivariate_linear_regression}
\end{equation}

Where:
\begin{itemize}
   \setlength{\itemsep}{0pt}%
   \setlength{\parskip}{0pt}%
   \setlength{\parsep}{0pt}%
      \item $\beta_3$ is the coefficient of the squared distance to the nearest subway station
      \item $\beta_4$ is the coefficient on the age of the house
      \item $\beta_5$ is a matrix of coefficients for district dummies
      \item $\beta_6$ is the coefficient of the year
      \item $\beta_7$ is a matrix of coefficients for the interaction of district and year dummies
      \item $\text{oldnew}$ is a dummy variable for the age of the house
      \item $\text{district}$ represents dummy variables for the district of the house
      \item $\text{year}$ represents dummy variables for the year of the observation
   \end{itemize}

We include a squared term for distance in our multivariate regression to account for the non-linear relationship between distance and price we find in Figure\ref{log_price_scatter}.

We now include controls for the age of the house, the district of the house, and the year of the observation. The economic basis for this, is we expect there to be 
price variation between older and newer homes because of the quality of the home. We also expect price variation between districts as some districts
hold premiums due to proximity to a city center or other amenities. We also expect price variation over time as the economy grows and inflation occurs.
Consider that some districts receive more or less government funding and expansion. Since we might expect the time varying effect to be different across districts, we include an interaction term between district and year.

\subsubsection{Multivariate Results}
Table~\ref{tab:summary_regression_table} hides the coefficients of the controls, but a full table of the results can be found in the appendix under Table~\ref{tab:full_MV_regression_results} and Table~\ref{tab:full_MV_regression_results_w_dummy}.

In our multivariate regression with the continuous polynomial for distance (Specification \ref{eq:multivariate_linear_regression}), we find that the coefficient is still economically insignificant (-0.0001261). As expected, controlling for district and time-varying effects increases the magnitude of our result, 
but we still conclude a precise null marginal effect of distance on house price.

In our final specification (Specification~\ref{eq:dummy_multivariate_linear_regression}),
 we find that the coefficient of the dummy variable for being close to a subway station is still economically and statistically significant. 
 However, much of the variation is now explained by the controls, and we find a more reasonable 0.18\% increase attributable to being close (Within 2km) to a subway station.
 This aligns with the results of \citet{rietveld_2007}, which finds that the price effect of subway stations is primarily in short distances.

Notably, our $R^2$ jumps from 0.03 in our univariate regression to 0.6 in our multivariate regression(Table \ref{tab:full_simple_regression_results} and Table~\ref{tab:full_MV_regression_results}).
This is a good sign that our multivariate specifications which include controls are explaining much of the variation in logarithmic house prices.

Looking at the predicted versus actual plots in Figure~\ref{MV_bucketed_predicted_vs_actual} and Figure~\ref{MV_dummy_bucketed_predicted_vs_actual}, we see that our models are well-specified, apart from at the lower tail.
Most predicted values lie on the 45 degree line, however, we are systematically underpredicting the prices of the lowest priced homes. This is further reflected in the residual fits
as seen in Figure \ref{MV_fitted_vs_residuals} and Figure \ref{MV_dummy_fitted_vs_residuals}. We see that our residuals are randomly distributed around 0 apart from at the lower tail, where they are systematically positive.

  \section{Discussion}
\label{sec:conclusion}
\subsection{Limitations of Results}
The analysis of our limitations can be broken into internal and external validity.

\subsubsection{Internal Validity}
 Our regression is misspecified and has an omitted variable bias.
  One potential omitted variable could be house size.
   We would expect house size to explain some of the variation in house price since larger houses command higher prices.
    However without further analysis it would be difficult to estimate the effect of house size since larger houses are in less dense areas of the city where property is also generally cheaper.
     We also run into the problem of simultaneous causality, areas that become trendy and develop would command higher housing prices, 
     which could justify a new subway station nearby. We attempt to control for this using time-invariant district fixed effects 
     but that does not solve the time-variant relationship where growing neighborhoods get subway allocations.
      In a future work we would frame a subway being built as a treatment, and attempt a difference-in-difference analysis, 
      so we can gauge economic significance and price effects of subways without having to consider some of the other omitted variables.
In our analysis we also assumed that our data was i.i.d. However this may not be true as we 
saw in the 2008 financial crisis when many people sold houses at the same time, leading to a depreciation in 
house price, causing panic and further selling. This means that there are regimes where selling a house has a causal effect on another person selling a house.

\subsubsection{External Validity}
The external validity of our regression is limited. The population studied are homes in the Greater London Area, 
with observations from 1995 to 2024. We can use our model to make inferences on homes that fit within this sample space, 
however, expect pricing dynamics to be different between cities and regimes. For example, in smaller cities we might expect 
price variation to be independent of subway locations if there is sufficient coverage.


\subsection{Conclusion}
Houses closer to subway stations in the Greater London Area see higher prices than those further away. 
This analysis includes controls for time, district, and a polynomial fit for the distance and price relationship.
The relationship is likely piece wise, beyond 2km away from a station the distance from a station has little effect on price.
We would warn against assuming external validity, and exogenous factors like governance and housing density could affect inferences. 
Marginal distance from subway stations are statistically significant but economically insignificant in their relationship with house price.
Being 'close' to a subway station however, is statistically and economically significant, even when controlling for district, building age,
 and time effects. We are wary about causality given that our residuals have a non-zero mean. Given the 
 distribution of our residuals vs fitted the specification is missing a factor that explains the variation is lower priced home. This may be size, which we've discussed,
 or some other factor.

  \newpage
  \bibliographystyle{chicago}
  \bibliography{references}

  \newpage
  \section*{Appendix}
  \renewcommand{\thesection}{\Alph{section}}
  \setcounter{section}{0}
  \section{Appendix}
\label{sec:appendix}

\subsection{Regression Summary and Figures}
\begin{table}[ht]
    \centering
    \begin{tabular}{lcccc}
        \toprule
        \textbf{Coefficient} & \textbf{\ref{eq:univariate_linear_regression}} & \textbf{\ref{eq:dummy_univariate_linear_regression}} & \textbf{\ref{eq:multivariate_linear_regression}} & \textbf{\ref{eq:dummy_multivariate_linear_regression}} \\
        \midrule
        \text{Intercept} & 12.52 & 12.22 &  14.57 & 14.57 \\
                           & (0.00208) & (0.00202) & (1.826) & (1.826) \\
        \text{min\_dist} & -0.0000441 & & -0.000126 &  \\
                           & (0.000000470) &  & (0.00000145) &  \\
        \text{close}      &  & 0.314 &  & 0.189 \\
                           &  & (0.00283) &  & (0.00290) \\
        \text{min\_dist$^2$} & & & 1.09e-08 & \\
                              &  & & (1.35e-10) &  \\
        \midrule
        \textbf{Controls} & No & No & Yes & Yes \\
        \bottomrule
    \end{tabular}
    \label{tab:summary_regression_table}
    \caption{Regression Coefficients and Standard Errors for Model Specifications 1-4. Controls hold district, year, house age, and year-district interactions.}
\end{table}

% Figure showing log_price scatter (../figs/log_price_distance_scatter.png). Label: log_price_scatter
\begin{figure}
    \centering
    \includegraphics[width=0.8\textwidth]{../figs/log_price_distance_scatter.png}
    \caption{Scatter plot of log price against distance to the nearest subway station. Notice the non-linear relationship, and piece wise nature beyond 2km.}
    \label{log_price_scatter}
\end{figure}

% Figure showing fitted vs residuals for multivariate regression (../figs/MV_predicted_vs_actual.png). Label: MV_predicted_vs_actual
\begin{figure}
    \centering
    \includegraphics[width=0.8\textwidth]{../figs/MV_fitted_vs_residuals.png}
    \caption{Fitted vs Residuals plot for the multivariate regression. The residuals are distributed randomly around 0 apart from at the lower tail where they break down.
    Results are after bucketing predicteds into bins for easier interpretation.
    Results are from the regression in Specification \ref{eq:multivariate_linear_regression}.}
    \label{MV_fitted_vs_residuals}
\end{figure}

% Figure showing expected vs actual for multivariate regression (../figs/MV_bucketed_predicted_vs_actual.png). Label: MV_bucketed_predicted_vs_actual
\begin{figure}
    \centering
    \includegraphics[width=0.8\textwidth]{../figs/MV_bucketed_expected_vs_actual.png}
    \caption{Bucketed Expected vs Actual plot for the multivariate regression. The expected values are bucketed into 10 bins and the actual values are averaged within each bin
    we do this becuase with over 340000 observations the plots otherwise are difficult to interpret. 
    Results are from the regression in Specification \ref{eq:multivariate_linear_regression}.}
    \label{MV_bucketed_predicted_vs_actual}
\end{figure}

% Same two but incl dummy
% Figure showing fitted vs residuals for multivariate regression with dummy (../figs/bucketed_dummy_expected_vs_actual.png). Label: MV_dummy_bucketed_predicted_vs_actual
\begin{figure}
    \centering
    \includegraphics[width=0.8\textwidth]{../figs/bucketed_dummy_expected_vs_actual.png}
    \caption{Bucketed Expected vs Actual plot for the multivariate regression with dummy. The expected values are bucketed into 10 bins and the actual values are averaged within each bin
    we do this because with over 340000 observations the plots otherwise are difficult to interpret.
    Results are from the regression in Specification \ref{eq:dummy_multivariate_linear_regression}.}
    \label{MV_dummy_bucketed_predicted_vs_actual}
\end{figure}

% Figure showing fitted vs residuals for multivariate regression with dummy (../figs/dummy_fitted_vs_residuals.png). Label: MV_dummy_fitted_vs_residuals
\begin{figure}
    \centering
    \includegraphics[width=0.8\textwidth]{../figs/dummy_fitted_vs_residuals.png}
    \caption{Fitted vs Residuals plot for the multivariate regression with dummy. The residuals are distributed randomly around 0 apart from at the lower tail where they break down.
    Results are after bucketing predicteds into bins for easier interpretation.
    Results are from the regression in Specification \ref{eq:dummy_multivariate_linear_regression}.}
    \label{MV_dummy_fitted_vs_residuals}
\end{figure}


\newpage
\subsection{Extensive Regression Results}
% Import simple regression table (../tables/simple_regression_table.tex). Label is full_simple_regression_results
{
\def\sym#1{\ifmmode^{#1}\else\(^{#1}\)\fi}
\begin{longtable}{l*{1}{c}}
\caption{Regression Results}\\
\toprule\endfirsthead\midrule\endhead\midrule\endfoot\endlastfoot
                    &\multicolumn{1}{c}{(1)}\\
                    &\multicolumn{1}{c}{log\_price}\\
\midrule
Min\_dist            &  -0.0000441\sym{***}\\
                    &(0.000000470)         \\
\addlinespace
Constant            &       12.52\sym{***}\\
                    &   (0.00208)         \\
\midrule
R-squared           &      0.0225         \\
Observations        &      347373         \\
\bottomrule
\multicolumn{2}{l}{\footnotesize Standard errors in parentheses}\\
\multicolumn{2}{l}{\footnotesize \sym{*} \(p<0.05\), \sym{**} \(p<0.01\), \sym{***} \(p<0.001\)}\\
\end{longtable}
\label{tab:full_simple_regression_results}
}

\label{tab:full_simple_regression_results}

% Import simple regression table with dummy (../tables/simple_regression_table_w_dummy.tex). Label is full_simple_regression_results_w_dummy
{
\def\sym#1{\ifmmode^{#1}\else\(^{#1}\)\fi}
\begin{longtable}{l*{1}{c}}
\caption{Regression results for the close/far dummy variable. Shows us the relationship is economically piece wise.}\\
\toprule\endfirsthead\midrule\endhead\midrule\endfoot\endlastfoot
                    &\multicolumn{1}{c}{(1)}\\
                    &\multicolumn{1}{c}{log\_price}\\
\midrule
Min\_dist            &       0.314\sym{***}\\
                    &   (0.00283)         \\
\addlinespace
Constant            &       12.22\sym{***}\\
                    &   (0.00202)         \\
\midrule
R-squared           &      0.0326         \\
Observations        &      347373         \\
\bottomrule
\multicolumn{2}{l}{\footnotesize Standard errors in parentheses}\\
\multicolumn{2}{l}{\footnotesize \sym{*} \(p<0.05\), \sym{**} \(p<0.01\), \sym{***} \(p<0.001\)}\\
\end{longtable}
\label{tab:full_simple_regression_results_w_dummy}
}

\label{tab:full_simple_regression_results_w_dummy}

% Import full multivariate regression table (../tables/full_MV_regression_results.tex)
{
\def\sym#1{\ifmmode^{#1}\else\(^{#1}\)\fi}
\begin{longtable}{l*{1}{c}}
\caption{Multivariate regression results, min\_dist is a polynomial and in-context with district, year, house age, and district-year interaction fixed effects.}\\
\toprule\endfirsthead\midrule\endhead\midrule\endfoot\endlastfoot
                    &\multicolumn{1}{c}{(1)}\\
                    &\multicolumn{1}{c}{log\_price}\\
\midrule
Min\_dist            &   -0.000126\sym{***}\\
                    &(0.00000145)         \\
\addlinespace
min\_dist\_squared    &    1.09e-08\sym{***}\\
                    &  (1.35e-10)         \\
\addlinespace
N                   &           0         \\
                    &         (.)         \\
\addlinespace
Y                   &       0.115\sym{***}\\
                    &   (0.00475)         \\
\addlinespace
Year=1995           &           0         \\
                    &         (.)         \\
\addlinespace
Year=1996           &      0.0600\sym{***}\\
                    &   (0.00653)         \\
\addlinespace
Year=1997           &       0.192\sym{***}\\
                    &   (0.00648)         \\
\addlinespace
Year=1998           &       0.320\sym{***}\\
                    &   (0.00663)         \\
\addlinespace
Year=1999           &       0.483\sym{***}\\
                    &   (0.00652)         \\
\addlinespace
Year=2000           &       0.662\sym{***}\\
                    &   (0.00685)         \\
\addlinespace
Year=2001           &       0.783\sym{***}\\
                    &   (0.00684)         \\
\addlinespace
Year=2002           &       0.940\sym{***}\\
                    &   (0.00698)         \\
\addlinespace
Year=2003           &       1.061\sym{***}\\
                    &   (0.00729)         \\
\addlinespace
Year=2004           &       1.136\sym{***}\\
                    &   (0.00756)         \\
\addlinespace
Year=2005           &       1.183\sym{***}\\
                    &   (0.00804)         \\
\addlinespace
Year=2006           &       1.255\sym{***}\\
                    &   (0.00826)         \\
\addlinespace
Year=2007           &       1.381\sym{***}\\
                    &   (0.00870)         \\
\addlinespace
Year=2008           &       1.383\sym{***}\\
                    &   (0.00997)         \\
\addlinespace
Year=2009           &       1.358\sym{***}\\
                    &    (0.0106)         \\
\addlinespace
Year=2010           &       1.435\sym{***}\\
                    &    (0.0108)         \\
\addlinespace
Year=2011           &       1.454\sym{***}\\
                    &    (0.0113)         \\
\addlinespace
Year=2012           &       1.501\sym{***}\\
                    &    (0.0116)         \\
\addlinespace
Year=2013           &       1.552\sym{***}\\
                    &    (0.0118)         \\
\addlinespace
Year=2014           &       1.672\sym{***}\\
                    &    (0.0123)         \\
\addlinespace
Year=2015           &       1.770\sym{***}\\
                    &    (0.0127)         \\
\addlinespace
Year=2016           &       1.844\sym{***}\\
                    &    (0.0134)         \\
\addlinespace
Year=2017           &       1.874\sym{***}\\
                    &    (0.0143)         \\
\addlinespace
Year=2018           &       1.895\sym{***}\\
                    &    (0.0147)         \\
\addlinespace
Year=2019           &       1.899\sym{***}\\
                    &    (0.0154)         \\
\addlinespace
Year=2020           &       1.954\sym{***}\\
                    &    (0.0158)         \\
\addlinespace
Year=2021           &       1.975\sym{***}\\
                    &    (0.0154)         \\
\addlinespace
Year=2022           &       2.032\sym{***}\\
                    &    (0.0164)         \\
\addlinespace
Year=2023           &       2.006\sym{***}\\
                    &    (0.0171)         \\
\addlinespace
Year=2024           &       2.001\sym{***}\\
                    &    (0.0189)         \\
\addlinespace
BARKING AND DAGENHAM&           0         \\
                    &         (.)         \\
\addlinespace
BARNET              &       2.765         \\
                    &     (1.732)         \\
\addlinespace
BEXLEY              &       5.589\sym{***}\\
                    &     (1.409)         \\
\addlinespace
BRENT               &      -18.80\sym{***}\\
                    &     (1.985)         \\
\addlinespace
BROMLEY             &       5.656\sym{***}\\
                    &     (1.398)         \\
\addlinespace
CAMDEN              &      -6.014\sym{**} \\
                    &     (2.238)         \\
\addlinespace
CITY OF LONDON      &      -23.00         \\
                    &     (15.16)         \\
\addlinespace
CITY OF WESTMINSTER &      -17.20\sym{***}\\
                    &     (2.563)         \\
\addlinespace
CROYDON             &       5.677\sym{***}\\
                    &     (1.403)         \\
\addlinespace
EALING              &      -1.319         \\
                    &     (1.711)         \\
\addlinespace
ENFIELD             &       2.589         \\
                    &     (1.548)         \\
\addlinespace
GREENWICH           &      -3.600\sym{*}  \\
                    &     (1.594)         \\
\addlinespace
HACKNEY             &      -29.40\sym{***}\\
                    &     (1.884)         \\
\addlinespace
HAMMERSMITH AND FULHAM&      -4.303\sym{*}  \\
                    &     (2.087)         \\
\addlinespace
HARINGEY            &      -11.89\sym{***}\\
                    &     (1.858)         \\
\addlinespace
HARROW              &       5.828\sym{**} \\
                    &     (1.884)         \\
\addlinespace
HAVERING            &       11.91\sym{***}\\
                    &     (1.403)         \\
\addlinespace
HILLINGDON          &       7.313\sym{***}\\
                    &     (1.485)         \\
\addlinespace
HOUNSLOW            &       5.766\sym{**} \\
                    &     (1.857)         \\
\addlinespace
ISLINGTON           &      -5.135\sym{*}  \\
                    &     (2.103)         \\
\addlinespace
KENSINGTON AND CHELSEA&      -12.13\sym{***}\\
                    &     (2.716)         \\
\addlinespace
KINGSTON UPON THAMES&       4.581\sym{**} \\
                    &     (1.757)         \\
\addlinespace
LAMBETH             &      -9.541\sym{***}\\
                    &     (1.563)         \\
\addlinespace
LEWISHAM            &      -17.67\sym{***}\\
                    &     (1.465)         \\
\addlinespace
MERTON              &      -0.689         \\
                    &     (1.881)         \\
\addlinespace
NEWHAM              &      -13.05\sym{***}\\
                    &     (1.550)         \\
\addlinespace
REDBRIDGE           &       1.208         \\
                    &     (1.579)         \\
\addlinespace
RICHMOND UPON THAMES&       9.189\sym{***}\\
                    &     (1.756)         \\
\addlinespace
SOUTHWARK           &      -17.08\sym{***}\\
                    &     (1.809)         \\
\addlinespace
SUTTON              &       8.695\sym{***}\\
                    &     (1.509)         \\
\addlinespace
TOWER HAMLETS       &       2.964         \\
                    &     (1.962)         \\
\addlinespace
WALTHAM FOREST      &      -22.81\sym{***}\\
                    &     (1.452)         \\
\addlinespace
WANDSWORTH          &      -2.433         \\
                    &     (1.529)         \\
\addlinespace
BARKING AND DAGENHAM $\times$ Year&    -0.00161\sym{*}  \\
                    &  (0.000762)         \\
\addlinespace
BARNET $\times$ Year&    -0.00268\sym{**} \\
                    &  (0.000863)         \\
\addlinespace
BEXLEY $\times$ Year&    -0.00424\sym{***}\\
                    &  (0.000702)         \\
\addlinespace
BRENT $\times$ Year &     0.00799\sym{***}\\
                    &  (0.000990)         \\
\addlinespace
BROMLEY $\times$ Year&    -0.00416\sym{***}\\
                    &  (0.000696)         \\
\addlinespace
CAMDEN $\times$ Year&     0.00186         \\
                    &   (0.00111)         \\
\addlinespace
CITY OF LONDON $\times$ Year&      0.0102         \\
                    &   (0.00756)         \\
\addlinespace
CITY OF WESTMINSTER $\times$ Year&     0.00753\sym{***}\\
                    &   (0.00128)         \\
\addlinespace
CROYDON $\times$ Year&    -0.00424\sym{***}\\
                    &  (0.000699)         \\
\addlinespace
EALING $\times$ Year&   -0.000691         \\
                    &  (0.000852)         \\
\addlinespace
ENFIELD $\times$ Year&    -0.00268\sym{***}\\
                    &  (0.000771)         \\
\addlinespace
GREENWICH $\times$ Year&    0.000413         \\
                    &  (0.000795)         \\
\addlinespace
HACKNEY $\times$ Year&      0.0133\sym{***}\\
                    &  (0.000939)         \\
\addlinespace
HAMMERSMITH AND FULHAM $\times$ Year&    0.000992         \\
                    &   (0.00104)         \\
\addlinespace
HARINGEY $\times$ Year&     0.00456\sym{***}\\
                    &  (0.000926)         \\
\addlinespace
HARROW $\times$ Year&    -0.00429\sym{***}\\
                    &  (0.000938)         \\
\addlinespace
HAVERING $\times$ Year&    -0.00737\sym{***}\\
                    &  (0.000699)         \\
\addlinespace
HILLINGDON $\times$ Year&    -0.00506\sym{***}\\
                    &  (0.000740)         \\
\addlinespace
HOUNSLOW $\times$ Year&    -0.00425\sym{***}\\
                    &  (0.000926)         \\
\addlinespace
ISLINGTON $\times$ Year&     0.00134         \\
                    &   (0.00105)         \\
\addlinespace
KENSINGTON AND CHELSEA $\times$ Year&     0.00515\sym{***}\\
                    &   (0.00135)         \\
\addlinespace
KINGSTON UPON THAMES $\times$ Year&    -0.00350\sym{***}\\
                    &  (0.000875)         \\
\addlinespace
LAMBETH $\times$ Year&     0.00345\sym{***}\\
                    &  (0.000778)         \\
\addlinespace
LEWISHAM $\times$ Year&     0.00744\sym{***}\\
                    &  (0.000730)         \\
\addlinespace
MERTON $\times$ Year&   -0.000965         \\
                    &  (0.000937)         \\
\addlinespace
NEWHAM $\times$ Year&     0.00496\sym{***}\\
                    &  (0.000772)         \\
\addlinespace
REDBRIDGE $\times$ Year&    -0.00205\sym{**} \\
                    &  (0.000787)         \\
\addlinespace
RICHMOND UPON THAMES $\times$ Year&    -0.00569\sym{***}\\
                    &  (0.000874)         \\
\addlinespace
SOUTHWARK $\times$ Year&     0.00720\sym{***}\\
                    &  (0.000902)         \\
\addlinespace
SUTTON $\times$ Year&    -0.00568\sym{***}\\
                    &  (0.000752)         \\
\addlinespace
TOWER HAMLETS $\times$ Year&    -0.00282\sym{**} \\
                    &  (0.000979)         \\
\addlinespace
WALTHAM FOREST $\times$ Year&     0.00988\sym{***}\\
                    &  (0.000723)         \\
\addlinespace
WANDSWORTH $\times$ Year&           0         \\
                    &         (.)         \\
\addlinespace
Constant            &       14.07\sym{***}\\
                    &     (1.525)         \\
\midrule
R-squared           &       0.633         \\
Observations        &      347373         \\
\bottomrule
\multicolumn{2}{l}{\footnotesize Standard errors in parentheses}\\
\multicolumn{2}{l}{\footnotesize \sym{*} \(p<0.05\), \sym{**} \(p<0.01\), \sym{***} \(p<0.001\)}\\
\end{longtable}
}

\label{tab:full_MV_regression_results}

% Import full multivariate regression table with dummy (../tables/full_MV_regression_results_w_dummy.tex)
{
\def\sym#1{\ifmmode^{#1}\else\(^{#1}\)\fi}
\begin{longtable}{l*{1}{c}}
\caption{Regression results using the close/far dummy Min\_dist is either 0 or 1. Shows us that the dummy effect is still economically and statistically significant in-context with our controls.}\\
\toprule\endfirsthead\midrule\endhead\midrule\endfoot\endlastfoot
                    &\multicolumn{1}{c}{(1)}\\
                    &\multicolumn{1}{c}{log\_price}\\
\midrule
Min\_dist=0          &           0         \\
                    &         (.)         \\
\addlinespace
Min\_dist=1          &       0.189\sym{***}\\
                    &   (0.00290)         \\
\addlinespace
N                   &           0         \\
                    &         (.)         \\
\addlinespace
Y                   &       0.119\sym{***}\\
                    &   (0.00476)         \\
\addlinespace
Year=1995           &           0         \\
                    &         (.)         \\
\addlinespace
Year=1996           &      0.0605\sym{***}\\
                    &   (0.00663)         \\
\addlinespace
Year=1997           &       0.192\sym{***}\\
                    &   (0.00649)         \\
\addlinespace
Year=1998           &       0.321\sym{***}\\
                    &   (0.00663)         \\
\addlinespace
Year=1999           &       0.482\sym{***}\\
                    &   (0.00655)         \\
\addlinespace
Year=2000           &       0.661\sym{***}\\
                    &   (0.00681)         \\
\addlinespace
Year=2001           &       0.782\sym{***}\\
                    &   (0.00684)         \\
\addlinespace
Year=2002           &       0.939\sym{***}\\
                    &   (0.00696)         \\
\addlinespace
Year=2003           &       1.060\sym{***}\\
                    &   (0.00732)         \\
\addlinespace
Year=2004           &       1.135\sym{***}\\
                    &   (0.00752)         \\
\addlinespace
Year=2005           &       1.182\sym{***}\\
                    &   (0.00793)         \\
\addlinespace
Year=2006           &       1.255\sym{***}\\
                    &   (0.00795)         \\
\addlinespace
Year=2007           &       1.380\sym{***}\\
                    &   (0.00827)         \\
\addlinespace
Year=2008           &       1.384\sym{***}\\
                    &   (0.00966)         \\
\addlinespace
Year=2009           &       1.360\sym{***}\\
                    &    (0.0100)         \\
\addlinespace
Year=2010           &       1.438\sym{***}\\
                    &   (0.00992)         \\
\addlinespace
Year=2011           &       1.457\sym{***}\\
                    &    (0.0103)         \\
\addlinespace
Year=2012           &       1.502\sym{***}\\
                    &    (0.0105)         \\
\addlinespace
Year=2013           &       1.554\sym{***}\\
                    &    (0.0106)         \\
\addlinespace
Year=2014           &       1.673\sym{***}\\
                    &    (0.0109)         \\
\addlinespace
Year=2015           &       1.771\sym{***}\\
                    &    (0.0113)         \\
\addlinespace
Year=2016           &       1.846\sym{***}\\
                    &    (0.0117)         \\
\addlinespace
Year=2017           &       1.877\sym{***}\\
                    &    (0.0122)         \\
\addlinespace
Year=2018           &       1.899\sym{***}\\
                    &    (0.0126)         \\
\addlinespace
Year=2019           &       1.903\sym{***}\\
                    &    (0.0130)         \\
\addlinespace
Year=2020           &       1.957\sym{***}\\
                    &    (0.0134)         \\
\addlinespace
Year=2021           &       1.979\sym{***}\\
                    &    (0.0133)         \\
\addlinespace
Year=2022           &       2.035\sym{***}\\
                    &    (0.0139)         \\
\addlinespace
Year=2023           &       2.010\sym{***}\\
                    &    (0.0146)         \\
\addlinespace
Year=2024           &       2.005\sym{***}\\
                    &    (0.0166)         \\
\addlinespace
BARKING AND DAGENHAM&           0         \\
                    &         (.)         \\
\addlinespace
BARNET              &       2.225         \\
                    &     (1.945)         \\
\addlinespace
BEXLEY              &       5.123\sym{**} \\
                    &     (1.974)         \\
\addlinespace
BRENT               &      -19.32\sym{***}\\
                    &     (2.103)         \\
\addlinespace
BROMLEY             &       5.577\sym{**} \\
                    &     (1.858)         \\
\addlinespace
CAMDEN              &      -6.402\sym{**} \\
                    &     (2.130)         \\
\addlinespace
CITY OF LONDON      &      -23.19\sym{**} \\
                    &     (7.644)         \\
\addlinespace
CITY OF WESTMINSTER &      -17.68\sym{***}\\
                    &     (2.043)         \\
\addlinespace
CROYDON             &       4.450\sym{*}  \\
                    &     (1.851)         \\
\addlinespace
EALING              &      -2.391         \\
                    &     (1.985)         \\
\addlinespace
ENFIELD             &       2.017         \\
                    &     (1.944)         \\
\addlinespace
GREENWICH           &      -4.107\sym{*}  \\
                    &     (2.007)         \\
\addlinespace
HACKNEY             &      -29.34\sym{***}\\
                    &     (2.163)         \\
\addlinespace
HAMMERSMITH AND FULHAM&      -4.642\sym{*}  \\
                    &     (2.135)         \\
\addlinespace
HARINGEY            &      -12.52\sym{***}\\
                    &     (2.030)         \\
\addlinespace
HARROW              &       5.490\sym{*}  \\
                    &     (2.198)         \\
\addlinespace
HAVERING            &       10.56\sym{***}\\
                    &     (1.962)         \\
\addlinespace
HILLINGDON          &       6.397\sym{**} \\
                    &     (1.954)         \\
\addlinespace
HOUNSLOW            &       5.913\sym{**} \\
                    &     (2.080)         \\
\addlinespace
ISLINGTON           &      -5.364\sym{*}  \\
                    &     (2.195)         \\
\addlinespace
KENSINGTON AND CHELSEA&      -12.54\sym{***}\\
                    &     (2.152)         \\
\addlinespace
KINGSTON UPON THAMES&       3.984         \\
                    &     (2.123)         \\
\addlinespace
LAMBETH             &      -9.866\sym{***}\\
                    &     (1.897)         \\
\addlinespace
LEWISHAM            &      -17.46\sym{***}\\
                    &     (1.927)         \\
\addlinespace
MERTON              &      -1.193         \\
                    &     (2.029)         \\
\addlinespace
NEWHAM              &      -14.91\sym{***}\\
                    &     (2.059)         \\
\addlinespace
REDBRIDGE           &      -0.242         \\
                    &     (2.023)         \\
\addlinespace
RICHMOND UPON THAMES&       8.947\sym{***}\\
                    &     (1.985)         \\
\addlinespace
SOUTHWARK           &      -18.27\sym{***}\\
                    &     (2.031)         \\
\addlinespace
SUTTON              &       8.082\sym{***}\\
                    &     (2.017)         \\
\addlinespace
TOWER HAMLETS       &       1.581         \\
                    &     (2.261)         \\
\addlinespace
WALTHAM FOREST      &      -23.41\sym{***}\\
                    &     (1.948)         \\
\addlinespace
WANDSWORTH          &      -3.230         \\
                    &     (1.829)         \\
\addlinespace
BARKING AND DAGENHAM $\times$ Year&    -0.00201\sym{*}  \\
                    &  (0.000911)         \\
\addlinespace
BARNET $\times$ Year&    -0.00281\sym{***}\\
                    &  (0.000719)         \\
\addlinespace
BEXLEY $\times$ Year&    -0.00439\sym{***}\\
                    &  (0.000738)         \\
\addlinespace
BRENT $\times$ Year &     0.00788\sym{***}\\
                    &  (0.000822)         \\
\addlinespace
BROMLEY $\times$ Year&    -0.00447\sym{***}\\
                    &  (0.000659)         \\
\addlinespace
CAMDEN $\times$ Year&     0.00168\sym{*}  \\
                    &  (0.000839)         \\
\addlinespace
CITY OF LONDON $\times$ Year&     0.00998\sym{**} \\
                    &   (0.00375)         \\
\addlinespace
CITY OF WESTMINSTER $\times$ Year&     0.00740\sym{***}\\
                    &  (0.000783)         \\
\addlinespace
CROYDON $\times$ Year&    -0.00402\sym{***}\\
                    &  (0.000654)         \\
\addlinespace
EALING $\times$ Year&   -0.000548         \\
                    &  (0.000746)         \\
\addlinespace
ENFIELD $\times$ Year&    -0.00280\sym{***}\\
                    &  (0.000718)         \\
\addlinespace
GREENWICH $\times$ Year&    0.000254         \\
                    &  (0.000760)         \\
\addlinespace
HACKNEY $\times$ Year&      0.0129\sym{***}\\
                    &  (0.000860)         \\
\addlinespace
HAMMERSMITH AND FULHAM $\times$ Year&    0.000783         \\
                    &  (0.000842)         \\
\addlinespace
HARINGEY $\times$ Year&     0.00448\sym{***}\\
                    &  (0.000775)         \\
\addlinespace
HARROW $\times$ Year&    -0.00451\sym{***}\\
                    &  (0.000882)         \\
\addlinespace
HAVERING $\times$ Year&    -0.00709\sym{***}\\
                    &  (0.000731)         \\
\addlinespace
HILLINGDON $\times$ Year&    -0.00500\sym{***}\\
                    &  (0.000726)         \\
\addlinespace
HOUNSLOW $\times$ Year&    -0.00472\sym{***}\\
                    &  (0.000808)         \\
\addlinespace
ISLINGTON $\times$ Year&     0.00107         \\
                    &  (0.000880)         \\
\addlinespace
KENSINGTON AND CHELSEA $\times$ Year&     0.00498\sym{***}\\
                    &  (0.000853)         \\
\addlinespace
KINGSTON UPON THAMES $\times$ Year&    -0.00362\sym{***}\\
                    &  (0.000834)         \\
\addlinespace
LAMBETH $\times$ Year&     0.00322\sym{***}\\
                    &  (0.000686)         \\
\addlinespace
LEWISHAM $\times$ Year&     0.00692\sym{***}\\
                    &  (0.000707)         \\
\addlinespace
MERTON $\times$ Year&    -0.00110         \\
                    &  (0.000774)         \\
\addlinespace
NEWHAM $\times$ Year&     0.00549\sym{***}\\
                    &  (0.000795)         \\
\addlinespace
REDBRIDGE $\times$ Year&    -0.00172\sym{*}  \\
                    &  (0.000771)         \\
\addlinespace
RICHMOND UPON THAMES $\times$ Year&    -0.00597\sym{***}\\
                    &  (0.000745)         \\
\addlinespace
SOUTHWARK $\times$ Year&     0.00740\sym{***}\\
                    &  (0.000776)         \\
\addlinespace
SUTTON $\times$ Year&    -0.00579\sym{***}\\
                    &  (0.000767)         \\
\addlinespace
TOWER HAMLETS $\times$ Year&    -0.00252\sym{**} \\
                    &  (0.000921)         \\
\addlinespace
WALTHAM FOREST $\times$ Year&     0.00979\sym{***}\\
                    &  (0.000721)         \\
\addlinespace
WANDSWORTH $\times$ Year&           0         \\
                    &         (.)         \\
\addlinespace
Constant            &       14.57\sym{***}\\
                    &     (1.826)         \\
\midrule
R-squared           &       0.631         \\
Observations        &      347373         \\
\bottomrule
\multicolumn{2}{l}{\footnotesize Standard errors in parentheses}\\
\multicolumn{2}{l}{\footnotesize \sym{*} \(p<0.05\), \sym{**} \(p<0.01\), \sym{***} \(p<0.001\)}\\
\end{longtable}
\label{tab:full_MV_regression_results_w_dummy}
}

\label{tab:full_MV_regression_results_w_dummy}
\end{document}