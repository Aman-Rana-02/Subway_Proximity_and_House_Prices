\section{Results}
\label{sec:results}
Our features are uncorrelated, the predicted vs actual fit is on the 45 degree line, and our errors are normal. 

In table 3 we increase the robustness of our analysis by adding controls for additional characteristics
 which likely biased our estimation from our simple linear regression. After holding characteristics regarding neighborhood,
  year, and whether the house was newly built or not, the significance of distance to a subway station on price decreases to
   -0.0000268. Due to the decrease in the magnitude  in the coefficient for subway distance, we concluded that these
    characteristics were causing a negative bias on our variable. We include the square of the distance as well,
     to account for non-linearity noticed, that houses further away from subway stations have a more negative
      relationship than a linear fit suggests.
We also see that holding these variables constant there were other independent variables which accounted for
 a much greater percentage of the difference in price. As expected the year the house was sold had a high
  significance with houses sold in 2024 being associated with a  2\% price increase compared to the mean house.
   This trend was similar with other years with houses being sold closer to the modern day being associated
    with greater price increases. This kind of relationship is expected as the price of real estate has tended
     to increase in the past 3 decades.
We also noted a higher significance relating to the neighborhood in which the house was sold than our original independent variable of closest subway station. Affluent neighborhoods such as Kensington were associated with a 1.5% increase in price compared to other houses sold. Alternatively newly built houses also had a higher significance of selling for .1% more than old houses. Overall this accounting of secondary explainer variables 
