\section{Discussion}
\label{sec:conclusion}
\subsection{Limitations of Results}
The analysis of our limitations can be broken into internal and external validity.

\subsubsection{Internal Validity}
 Our regression is misspecified and has an omitted variable bias.
  One potential omitted variable could be house size.
   We would expect house size to explain some of the variation in house price since larger houses command higher prices.
    However without further analysis it would be difficult to estimate the effect of house size since larger houses are in less dense areas of the city where property is also generally cheaper.
     We also run into the problem of simultaneous causality, areas that become trendy and develop would command higher housing prices, 
     which could justify a new subway station nearby. We attempt to control for this using time and district fixed effects, 
     but that does not solve the time-variant relationship where growing neighborhoods get subway allocations.
      In a future work we would frame a subway being built as a treatment, and attempt a difference-in-difference analysis, 
      so we can gauge economic significance and price effects of subways without having to consider some of the other omitted variables.
In our analysis we also assumed that our data was i.i.d. However, this may not be true as we 
saw in the 2008 financial crisis when many people sold houses at the same time, leading to a depreciation in 
house price, causing panic and further selling. This means that there are regimes where selling a house has a causal effect on another person selling a house.

\subsubsection{External Validity}
The external validity of our regression is limited. The population studied are homes in the Greater London Area, 
with observations from 1995 to 2024. We can use our model to make inferences on homes that fit within this sample space, 
however, expect pricing dynamics to be different between cities and regimes. For example, in smaller cities we might expect 
price variation to be independent of subway locations if there is sufficient coverage.


\subsection{Conclusion}
Houses closer to subway stations in the Greater London Area see higher prices than those further away. 
This analysis includes controls for time, district, and a polynomial fit for the distance and price relationship.
The relationship is likely piece wise, beyond 2km away from a station the distance from a station has little effect on price.
We would warn against assuming external validity, and exogenous factors like governance and housing density could affect inferences. 
Marginal distance from subway stations are statistically significant but economically insignificant in their relationship with house price.
Being `close' to a subway station however, is statistically and economically significant, even when controlling for district, building age,
 and time effects. We are wary about causality given that our residuals have a non-zero mean. Given the 
 distribution of our residuals vs fitted, the specification is missing a factor that explains the variation in lower priced homes. This may be size, which we've discussed,
 or some other factor.