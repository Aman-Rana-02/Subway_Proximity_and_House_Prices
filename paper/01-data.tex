\section{Data}
\label{sec:data}
We began our analysis by sourcing data from ONS, the national statistics arm of the UK government,
 which provided panel data containing each lease transfer (home sale) that has occurred in the U.K. from 1995 to today \citep{land_ukgov}.
  We pruned this dataset to only contain observations within the Greater London Area.
   This dataset amounts to approximately 350 thousand observations. Each unit of observation represents
    a single occurrence of a lease transfer in London containing the date, coordinates, neighborhood, and price of each observation's
     lease transfer.  The second source of our data was a table containing the coordinates of each subway station in the
      London underground and the date at which it was opened \citep{subway_wikipedia}. 
To begin our analysis we combined these two datasets by using the date and coordinates of our houses to search for the closest
 subway station to the current house in the second dataset. This allowed us to get the minimum distance to a subway station at
  the time of its sale which we added to our ONS dataset as the variable calculating the euclidean distance to the closest station
   and a variable for the coordinate of the closest subway. For all houses which were further than 10 km away from a subway station
    we set their distance to be equal to 10 km to account for outliers in our data.
Delving into the summary statistics of our data, we can look at how the data is spread
 across our different variables. For our main variables of interest, the distance to subway stations, we see that we have an
  average distance of 2821 meters with a variance of 2915 meters (Table~\ref{tab:continuous_summary_stats}). This means that on average houses are located quite far from
   a subway station but also contain vast amounts of variation. From our earlier work on the matter the reason for this high
    variation and mean is due to the fact that a lot of houses are located very close to subway stations but there is a very
     long tail in our data with a significant number of houses located over 5 kilometers away. We truncate these
     outliers at 10km, and the distribution of the minimum distance to a subway station can be seen in Figure~\ref{min_dist_distribution}.
     Raw house price is also right skewed with a long tail, so we take the log of house prices to make our regression more interpretable
      and the distribution of log price can be seen in Figure~\ref{log_price_distribution}.
On the categorical side of things we see that our houses are very evenly spread between our neighborhoods with the largest
 neighborhood only being 2 points off of the mean (Table~\ref{tab:summary_district}). Similarly, our lease transfers were also very evenly spread between years
  with each year accounting for approximately 2\% 
  of our dataset (Table~\ref{tab:summary_year}). This is good for the robustness of our future models since the even spread of data between 
  our neighbourhoods would provide us enough data to account for outliers within neighbourhoods. 
  On the other hand the spread between newly built and old homes is quite dismal with 97\%
   of lease transfers being accounted for by the sale of pre-owned homes and as a result it will be more sensitive to outliers (Table~\ref{tab:summary_ON}).
