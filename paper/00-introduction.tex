\section{Introduction}
The relationship between transport infrastructure and real estate prices has been a topic of interest in econometrics for many years now. Accessibility to public transit has been long established as a key factor influencing the price of residential properties, particularly in dense urban environments where vehicle mobility is greatly constrained.
 In the context of the Greater London Area subway stations are the most efficient method of transportation for most commuters and thereby play a critical role in shaping the greater housing market.
  To this end in this paper we investigate how the euclidean distance from a residence to its nearest subway station affects market price 

By focusing on the Greater London Area our study contributes to already established relationships between subway proximity and house prices. 
Early studies such as those done by \citet{muth_1969} are cornerstones in establishing a relationship between densification and land use in cities. 
More recently a greater number of studies have found a significant positive relationship between transit proximity and real estate prices 
in cities such as those in Shanghai \citep{zhou_2019} where subway stations proximity was specifically correlated to 
increased house prices at statistically significant results. 
However London presents unique opportunities for research. 
As the historical pioneer in subway technology the city has had 
the opportunity to develop around an already established subway system and culture. 
This study extends the findings of existing literature by using highly specific spatial data and 
leveraging advanced analysis techniques to create clear and insightful findings on the pricing dynamics of the London underground.


The rest of the paper is structured as follows: Section \ref{sec:data} describes the data used in this analysis. Section \ref{sec:regression_analysis} outlines the methodology and model specifications used to measure the subway effect.
Section \ref{sec:results} presents the results of our analysis. Section \ref{sec:conclusion} discusses the limitations of our methodology and concludes. Appendix \ref{sec:appendix} holds figures and tables.
